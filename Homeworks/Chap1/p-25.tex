\begin{problem}
设 $f : \N{} \to \N{}$ 定义为
\[
f(n) = \text{$\pi$ 的十进制展开式中第 $n$ 位数字}
\]
例如 $f(0)=3, f(1)=1,f(2)=4$, 证明: $f \in \GRF$.
\end{problem}

\begin{solution}
Leibniz 公式 $\frac{\pi}{4} = 1 - \frac{1}{3} + \frac{1}{5} - \frac{1}{7} + \frac{1}{9} - \dots$.

定义 $L(n,k)$ 表示 Leibniz 公式的前 $n+1$ 项和乘 $k$ 下取整的值, $L(n,k) = \sum_{i=0}^n (-1)^{i} \frac{k}{2i+1} = \big (\sum_{i=0}^n N(rs(i, 2)) \cdot \frac{k}{2i+1} \big) - \big(\sum_{i=0}^n rs(i, 2) \cdot \frac{k}{2i+1} \big) \in \EF$.

所以, $f(n) = rs(L(100^n, 4 \cdot 10^n), 10)$.
\end{solution}