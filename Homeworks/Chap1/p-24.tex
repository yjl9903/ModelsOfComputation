\begin{problem}
设 $g: \N{} \to \N{}$ 满足
\begin{align*}
g(0) & = 0 \\
g(1) & = 1 \\
g(n + 2) & = rs((2002g(n+1)+2003g(n)), 2005)
\end{align*}
(1) 试求 $g(2006)$; (2) 证明: $g \in \EF$.
\end{problem}

\begin{solution}
定义 $h : \N{} \to \N{}$ 满足
\begin{align*}
h(0) & = 0 \\
h(1) & = 1 \\
h(n + 2) & = 2002h(n+1)+2003h(n)
\end{align*}

首先有一个引理: 对于任意 $x \in \N{}$, $g(x) = rs(h(x), 2005)$.

使用数学归纳法证明.

奠基, $g(0) = rs(h(0), 2005) = 0$, $g(1) = rs(h(1), 2005) = 1$.

归纳假设, 对于 $x = k, k + 1$ 成立, $g(k) = rs(h(k), 2005)$, $g(k+1) = rs(h(k+1), 2005)$. 于是有
\begin{align*}
g(k+2) & = rs(2002g(k+1)+2003g(k), 2005) \\
& = rs(2002 \cdot rs(g(k+1), 2005) + 2003 \cdot rs(g(k), 2005), 2005) \\
& = rs(2002 \cdot h(k+1) + 2003 \cdot h(k), 2005) \\
& = rs(h(k+2), 2005)
\end{align*}
因此, 结论对于 $x = k + 2$ 也成立, 证毕.

使用生成函数技术, 可以计算出 $h(n) = \frac{(-1)^{n+1}+2003^n}{2004}$.

\begin{enumerate}
\item 问题也就是求 $h(2006) = \frac{2003^{2006}-1}{2004} \bmod 2005$.

% 首先, $2004 \equiv -1 ~ (\bmod ~ 2005)$, $2004^2 \equiv 1 ~ (\bmod ~ 2005)$. 于是, $h(2006) \equiv (2003^{2006}-1) \cdot 2004 ~ (\bmod ~ 2005)$.
首先, $2003 \equiv -2 (\bmod ~ 2005)$, $2003^{2006} \equiv 2^{2006} (\bmod ~ 2005)$.

根据费马小定理, 对于质数 $p$ 有 $a^{p-1} \equiv 1 (\bmod ~ p)$. 因为 $2005 = 5 \cdot 401$. 对于因子 $5$, $2006 \equiv 2 (\bmod ~ 4)$, 所以 $2^{2006} \equiv 2^2 \equiv 4 (\bmod ~ 5)$. 对于因子 $401$, $2006 \equiv 6 (\bmod ~ 400)$, 所以 $2^{2006} \equiv 2^6 \equiv 64 (\bmod ~ 401)$. 最后使用中国剩余定理, $2^{2006} \equiv 64 (\bmod ~ 2005)$.

所以 $h(2006) \equiv \frac{2003^{2006}-1}{2004} \equiv 2005 - (64 - 1) \equiv 1942 (\bmod ~ 2005)$, $g(2006) = 1942$.
\item 因为 $h(n) = \frac{(-1)^{n+1}+2003^n}{2004} \in \EF$, 所以 $g(n) = rs(h(n), 2005) \in \EF$.

也可以使用定理 1.31, 注意到 $g(x) \le 2005$, 所以 $g \in \EF$.
\end{enumerate}
\end{solution}