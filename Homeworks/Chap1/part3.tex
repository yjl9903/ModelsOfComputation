\newcommand{\Ack}{\operatorname{Ack}{}}

\begin{problem}
证明: $\Ack(4,n) \in \PRF - \EF$.
\end{problem}

\begin{solution}
1. 证明 $\Ack(4, n) \in \PRF$.

证明一个更强命题: 对于固定的 $i$, $f_i(n) = \Ack(i, n) \in \PRF$.

奠基: $\Ack(0, n) = n + 1 \in \PRF$.

归纳假设: 当 $i = k$ 时, $f_k(n) = \Ack(k, n) \in \PRF$.

递推: $\Ack(k+1,0) = \Ack(k, 1)$, $\Ack(k+1, n+1) = \Ack(k, \Ack(k + 1, n))$, 因此可以构造 $f_{k+1} = h$:
\begin{align*}
h(0) & = f_k(1) \\
h(n + 1) & = f_k(h(n))
\end{align*}

因为 $f_k \in \PRF$, 所以 $f_{k+1} = h \in \PRF$.

因此 $Ack(4,n) = f_4(n) \in \PRF$.

2. 证明 $\Ack(4, n) \not\in \EF$.

根据上述归纳过程, 我们可以计算出 $\Ack(1, n), \Ack(2, n), \Ack(3, n), \Ack(4, n)$ 的具体形式.
\begin{align*}
\Ack(1, n) & = 2 + n \\
\Ack(2, n) & = 2n + 3 \\
\Ack(3, n) & = 2^{n+3} - 3 \\
\Ack(4, n) & = \underbrace{2^{{.}^{{.}^{{.}^{2}}}}}_{n+3 ~ \text{个} ~ 2} - 3
\end{align*}
然后证明 $\Ack(4, n)$ 不能被 $G(k, n)$ 控制.

反证法. 假设 $\Ack(4, n)$ 被 $G$ 控制, 则存在 $k_0 \in \N{}$, 使得 $\forall n \in \N{}$, $\Ack(4, n) \le G(k_0, n) = \underbrace{2^{{.}^{{.}^{{.}^{2^n}}}}}_{k_0 ~ \text{个} ~ 2}$.

令 $n = k_0$, $\Ack(4, n) = \underbrace{2^{{.}^{{.}^{{.}^{2}}}}}_{k_0 + 3 ~ \text{个} ~ 2} - 3 > \underbrace{2^{{.}^{{.}^{{.}^{2^{k_0}}}}}}_{k_0 ~ \text{个} ~ 2}$, 矛盾.

因此 $\Ack(4, n) \not\in \EF$.

综上 $\Ack(4, ) \in \PRF - \EF$.
\end{solution}

\begin{problem}
设 $f : \N{} \to \N{}$, $f$ 为一一映射, 证明: $f \in \GRF \Leftrightarrow f^{-1} \in \GRF$,
\end{problem}

\begin{solution}
先证明充分性, $f \in \GRF \Rightarrow f^{-1} \in \GRF$. 构造
\[
f^{-1} = \mu_{y} [f(y) - x]
\]
因为 $f$ 是一一映射, 因此 $f(y) = x$ 的根 $y$ 必定是存在且唯一的.

对于必要性, 因为 $(f^{-1})^{-1} = f$ 且 $f^{-1}$ 也是一一映射, 用类似上述构造容易证明.
\end{solution}

\begin{problem}
设 $p(x)$ 为整系数多项式, 令 $f : \N{} \to \N{}$ 定义为 $f(a) = p(x) - a$ 对于 $x$ 的非负整数根, 证明: $f \in \RF$.
\end{problem}

\begin{solution}
令 $p(x) = b_n x^n + b_{n-1}x^{n-1} + \dots + b_0$, 令 $S = \{ i ~ | ~ b_i \ge 0 \}$, $T = \{ i ~ | ~ b_i < 0 \}$. 于是
\[
p(x)-a = \sum_{i \in S} b_i x^i - (a + \sum_{i \in T} |b_i| x^i) \in \EF
\]
因此 $f = \mu_{x} [p(x) - a] \in \RF$.
\end{solution}

\begin{problem}
设
\[
f(x,y) = \begin{cases}
x / y, & \text{若} ~ y \neq 0 ~ \text{且} ~ y \mid x \\
\uparrow, & \text{否则}
\end{cases}
\]
证明: $f \in \RF$.
\end{problem}

\begin{solution}
$f(x, y) = \mu_{k}. [x \dotdotminus k \times y] + \mu_{k}. [N(x + y)] \in \RF$.
\end{solution}

\begin{problem}
设 $g: \N{} \to \N{}$ 满足
\begin{align*}
g(0) & = 0 \\
g(1) & = 1 \\
g(n + 2) & = rs((2002g(n+1)+2003g(n)), 2005)
\end{align*}
(1) 试求 $g(2006)$; (2) 证明: $g \in \EF$.
\end{problem}

\begin{solution}
定义 $h : \N{} \to \N{}$ 满足
\begin{align*}
h(0) & = 0 \\
h(1) & = 1 \\
h(n + 2) & = 2002h(n+1)+2003h(n)
\end{align*}

首先有一个引理: 对于任意 $x \in \N{}$, $g(x) = rs(h(x), 2005)$.

使用数学归纳法证明.

奠基, $g(0) = rs(h(0), 2005) = 0$, $g(1) = rs(h(1), 2005) = 1$.

归纳假设, 对于 $x = k, k + 1$ 成立, $g(k) = rs(h(k), 2005)$, $g(k+1) = rs(h(k+1), 2005)$. 于是有
\begin{align*}
g(k+2) & = rs(2002g(k+1)+2003g(k), 2005) \\
& = rs(2002 \cdot rs(g(k+1), 2005) + 2003 \cdot rs(g(k), 2005), 2005) \\
& = rs(2002 \cdot h(k+1) + 2003 \cdot h(k), 2005) \\
& = rs(h(k+2), 2005)
\end{align*}
因此, 结论对于 $x = k + 2$ 也成立, 证毕.

使用生成函数技术, 可以计算出 $h(n) = \frac{(-1)^{n+1}+2003^n}{2004}$.

\begin{enumerate}
\item 问题也就是求 $h(2006) = \frac{2003^{2006}-1}{2004} \bmod 2005$.

% 首先, $2004 \equiv -1 ~ (\bmod ~ 2005)$, $2004^2 \equiv 1 ~ (\bmod ~ 2005)$. 于是, $h(2006) \equiv (2003^{2006}-1) \cdot 2004 ~ (\bmod ~ 2005)$.
首先, $2003 \equiv -2 (\bmod ~ 2005)$, $2003^{2006} \equiv 2^{2006} (\bmod ~ 2005)$.

根据费马小定理, 对于质数 $p$ 有 $a^{p-1} \equiv 1 (\bmod ~ p)$. 因为 $2005 = 5 \cdot 401$. 对于因子 $5$, $2006 \equiv 2 (\bmod ~ 4)$, 所以 $2^{2006} \equiv 2^2 \equiv 4 (\bmod ~ 5)$. 对于因子 $401$, $2006 \equiv 6 (\bmod ~ 400)$, 所以 $2^{2006} \equiv 2^6 \equiv 64 (\bmod ~ 401)$. 最后使用中国剩余定理, $2^{2006} \equiv 64 (\bmod ~ 2005)$.

所以 $h(2006) \equiv \frac{2003^{2006}-1}{2004} \equiv 2005 - (64 - 1) \equiv 1942 (\bmod ~ 2005)$, $g(2006) = 1942$.
\item 因为 $h(n) = \frac{(-1)^{n+1}+2003^n}{2004} \in \EF$, 所以 $g(n) = rs(h(n), 2005) \in \EF$.

也可以使用定理 1.31, 注意到 $g(x) \le 2005$, 所以 $g \in \EF$.
\end{enumerate}
\end{solution}

\begin{problem}
设 $f : \N{} \to \N{}$ 定义为
\[
f(n) = \text{$\pi$ 的十进制展开式中第 $n$ 位数字}
\]
例如 $f(0)=3, f(1)=1,f(2)=4$, 证明: $f \in \GRF$.
\end{problem}

\begin{solution}
Leibniz 公式 $\frac{\pi}{4} = 1 - \frac{1}{3} + \frac{1}{5} - \frac{1}{7} + \frac{1}{9} - \dots$.

定义 $L(n,k)$ 表示 Leibniz 公式的前 $n+1$ 项和乘 $k$ 下取整的值, $L(n,k) = \sum_{i=0}^n (-1)^{i} \frac{k}{2i+1} = \big (\sum_{i=0}^n N(rs(i, 2)) \cdot \frac{k}{2i+1} \big) - \big(\sum_{i=0}^n rs(i, 2) \cdot \frac{k}{2i+1} \big) \in \EF$.

所以, $f(n) = rs(L(100^n, 4 \cdot 10^n), 10)$.
\end{solution}
