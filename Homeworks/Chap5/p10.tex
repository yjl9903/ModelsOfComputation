\begin{problem}
设 $f : \N{} \to \N{}$ 定义如下:
\begin{align*}
f(0) & = 0 \\
f(x + 1) & = g(f(x))
\end{align*}
证明: 若 $g$ 为 Turing-可计算, 则 $f$ 为 Turing-可计算.
\end{problem}

\begin{solution}
因为 $g$ 为 Turing-可计算, 因此存在机器 \machine{$g$} 计算函数 $g$.

构造机器 $M_1$ 如表 \ref{sol:5.10-1}.

\begin{table}[H]
    \centering
    \begin{tabularx}{\textwidth}{@{}Y|Y|Y@{}} \hhline
          & 0   & 1   \\ \hline
        1 &     & 0R2 \\ \hline
        2 & 0Rv & 1R3 \\ \hline
        3 & 1R4 & 1R3 \\ \hhline
    \end{tabularx}
    \caption{解答 5.10 $M_1$}
    \label{sol:5.10-1}
\end{table}

所以, 若 $x = 0$ 则 $M_1 | 1 : 0\rwhead{\overline{x}}0\overline{y}0\cdots \twoheadrightarrow v : 000\rwhead{\overline{y}}0\cdots$; 

若 $x > 0$ 则 $M_1 | 1 : 0\rwhead{\overline{x}}0\overline{y}0\cdots \twoheadrightarrow 4 : 001^x0\rwhead{\overline{y}}0\cdots$.

令 $M_2$ 为 $M_1 \concat \machine{$g$} + 3 \concat \compressmachine \concat \shiftlmachine$, 从而:

若 $x = 0$ 则 $M_2 | 1 : 0\rwhead{\overline{x}}0\overline{y}0\cdots \twoheadrightarrow v : 000\rwhead{\overline{y}}0\cdots$;

若 $x > 0$ 则 $M_1 | 1 : 0\rwhead{\overline{x}}0\overline{y}0\cdots \twoheadrightarrow w : 00\rwhead{1^x}0\overline{g(y)}0\cdots$, 其中 $w$ 为 $M_2$ 的输出状态.

最后, 构造机器 $M_3$ 如表 \ref{sol:5.10-3}, 为输入后添加一个 $\overline{0}$.

\begin{table}[H]
    \centering
    \begin{tabularx}{\textwidth}{@{}Y|Y|Y@{}} \hhline
          & 0   & 1   \\ \hline
        1 & 0R2 & 1R1 \\ \hline
        2 & 1L3 &     \\ \hline
        3 & 0L4 &     \\ \hline
        3 & 0R5 & 1L4 \\ \hhline
    \end{tabularx}
    \caption{解答 5.10 $M_3$}
    \label{sol:5.10-3}
\end{table}

令机器 \machine{$f$} 为 $M_3 \concat \operatorname{repeat} M_2$, 即 $M_3 \concat M[w := 1]$ 计算函数 $f$, 因此 $f$ 为 Turing-可计算的.
\end{solution}
