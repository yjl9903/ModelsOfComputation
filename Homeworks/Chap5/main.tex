\documentclass[UTF8, 12pt, a4paper, oneside]{ctexart}
\usepackage[T1]{fontenc}
\usepackage{amsmath, amsthm, amssymb, bm, commath, color, framed, graphicx, hyperref, mathrsfs, tcolorbox, tabularx}
\usepackage{geometry}
\usepackage{float}
\geometry{a4paper,left=2.1cm,right=2.1cm,top=2.97cm,bottom=2.97cm}

\title{\textbf{计算理论作业3}}
\author{颜俊梁 MF21330103}
\date{\today}

\linespread{1.5}
\definecolor{shadecolor}{RGB}{255, 255, 255}
\newcounter{problemname}
\newenvironment{problem}{\begin{shaded}\stepcounter{problemname}\par\noindent\textbf{题目5.\arabic{problemname}. }}{\end{shaded}\par}
\newenvironment{solution}{\par\noindent\textbf{解答. }}{\par}
\newenvironment{note}{\par\noindent\textbf{题目\arabic{problemname}的注记. }}{\par}

\newcommand{\N}[1][]{\mathbb{N}^#1}
\newcommand{\BNF}{\beta\text{-nf}}
\newcommand{\lambdabstract}[2]{\lambda {#1}.\,{#2}}
\newcommand{\sreduceto}{\rightarrow_\beta}
\newcommand{\reduceto}{\twoheadrightarrow_\beta}
\newcommand{\numeral}[1]{\left\ulcorner #1 \right\urcorner}
\newcommand{\IF}{\mathcal{IF}}
\newcommand{\BF}{\mathcal{BF}}
\newcommand{\EF}{\mathcal{EF}}
\newcommand{\PRF}{\mathcal{PRF}}
\newcommand{\GRF}{\mathcal{GRF}}
\newcommand{\RF}{\mathcal{RF}}

\def\dotminus{\mathbin{\ooalign{\hss\raise1ex\hbox{.}\hss\cr
  \mathsurround=0pt$-$}}}
\def\dotdotminus{\mathbin{\ooalign{\hss\raise1ex\hbox{..}\hss\cr
  \mathsurround=0pt$-$}}}

\tcbset{colframe=black,colback=white,colupper=black,
fonttitle=\bfseries,nobeforeafter,center title,size=small}
\tcbset{tcbox raise base}
\newtcbox{\machine}{on line,
arc = 0pt, outer arc = 0pt,
colback = white!10!white, colframe = black!50!black,
boxsep = 0pt, left = 4pt, right = 4pt, top = 4pt, bottom = 4pt, boxrule = 1pt}

\makeatletter
\def\hlinewd#1{%
\noalign{\ifnum0=`}\fi\hrule \@height #1 %
\futurelet\reserved@a\@xhline}
\makeatother

\newcommand{\hhline}{\hlinewd{1pt}}
\newcommand{\concat}{\mapstochar\Rightarrow}
\newcolumntype{Y}{>{\centering\arraybackslash}X}

\newcommand{\rwhead}[1]{\underset{\uparrow}{#1}}
\newcommand{\copymachine}[1]{\machine{$\text{copy}_#1$}}
\newcommand{\compressmachine}{\machine{$\text{compress}$}}
\newcommand{\shiftlmachine}{\machine{$\text{shiftl}$}}
\newcommand{\shiftrmachine}{\machine{$\text{shiftr}$}}
\newcommand{\erasemachine}{\machine{$\text{erase}$}}
\newcommand{\pangle}[1]{\left \langle #1 \right \rangle}

\begin{document}

\maketitle

\begin{problem}
构造机器计算函数 $f(x,y,z)=y$.
\end{problem}

\begin{solution}
\begin{table}[H]
    \centering
    \begin{tabularx}{\textwidth}{@{}Y|Y|Y@{}} \hhline
          & 0   & 1   \\ \hline
        1 & 0R2 & 0R1 \\ \hline
        2 & 0R3 & 1R2 \\ \hline
        3 & 0L4 & 0R3 \\ \hline
        4 & 0L4 & 1L5 \\ \hline
        5 & 0R6 & 1L5 \\ \hhline
    \end{tabularx}
    \caption{解答 5.1}
    \label{sol:5.1}
\end{table}

对于该机器, 定义见表\ref{sol:5.1}, 输入 $1 : 0\underset{\uparrow}{1^{x+1}}01^{y+1}01^{z+1}0\cdots$, 输出 $6 : 0^{x+3}\underset{\uparrow}{1^{y+1}}0\cdots$.
\end{solution}

\begin{problem}
试求 $SSSS$ 的 $\beta$-nf.
\end{problem}

\begin{solution}
标准组合子 $S = \lambdabstract{xyz}{xz(yz)}$.
\begin{align*}
~ & S S S S \\
\to_{\beta} ~ & (\lambdabstract{xyz}{xz(yz)}) S S S \\
\to_{\beta} ~ & S S (S S) \\
= ~ & (\lambdabstract{xyz}{xz(yz)}) S (S S) \\
\to_{\beta} ~ & \lambdabstract{z}{S z (S S z)} \\
= ~ & \lambdabstract{z}{(\lambdabstract{xyz}{xz(yz)}) z (S S z)} \\
\to_{\beta} ~ & \lambdabstract{z}{\lambdabstract{l}{zl(S S z l)}} \\
= ~ & \lambdabstract{z}{\lambdabstract{l}{zl((\lambdabstract{xyz}{xz(yz)})Szl)}} \\
\to_{\beta} ~ & \lambdabstract{z}{\lambdabstract{l}{zl(Sl(zl))}} \\
= ~ & \lambdabstract{z}{\lambdabstract{l}{zl((\lambdabstract{xyz}{xz(yz)})l(zl))}} \\
\to_{\beta} ~ & \lambdabstract{z}{\lambdabstract{l}{zl(\lambdabstract{k}{lk(zlk)})}}
\end{align*}

\end{solution}

\begin{problem}
证明: $(\lambdabstract{x}{xxx}) (\lambdabstract{x}{xxx})$ 没有 $\BNF$.
\end{problem}

\begin{solution}
令 $W = (\lambdabstract{x}{xxx}) (\lambdabstract{x}{xxx})$.
\begin{flalign*}
W = ~ & (\lambdabstract{x}{xxx}) (\lambdabstract{x}{xxx}) \\
\to_\beta ~ & (\lambdabstract{x}{xxx}) (\lambdabstract{x}{xxx}) (\lambdabstract{x}{xxx}) \\
= ~ & W (\lambdabstract{x}{xxx}) \\
\to_\beta ~ & W (\lambdabstract{x}{xxx}) (\lambdabstract{x}{xxx}) \\
\to_\beta ~ & W (\lambdabstract{x}{xxx}) (\lambdabstract{x}{xxx}) (\lambdabstract{x}{xxx}) \\
\to_\beta ~ & \dots
\end{flalign*}
在每步规约时, 有且仅有一个 $\beta-\text{redex}$ ($W = (\lambdabstract{x}{xxx}) (\lambdabstract{x}{xxx})$), 对它进行规约后产生了一个无穷规约链, 因此 $(\lambdabstract{x}{xxx}) (\lambdabstract{x}{xxx})$ 没有 $\BNF$.
\end{solution}
\begin{problem}
构造机器计算函数 $f(x)=2^x$.
\end{problem}

\begin{solution}
对于该机器, 定义见表\ref{sol:5.4}, 输入 $1 : 0\underset{\uparrow}{1^{x+1}}0\cdots$, 输出 $27 : 0^{x+3}\underset{\uparrow}{1^{2^x+1}}0\cdots$.

\begin{table}[H]
    \centering
    \begin{tabularx}{\textwidth}{@{}Y|Y|Y@{}} \hhline
          & 0    & 1    \\ \hline
        1 & 0R2  & 1R1  \\ \hline
        2 & 1R3  &      \\ \hline
        3 & 1L4  &      \\ \hline
        4 &      & 1L5  \\ \hline
        5 & 0L6  &      \\ \hline
        6 & 0R7  & 1L6  \\ \hline
        7 &      & 0R8  \\ \hline
        8 & 0R27 & 1R9  \\ \hline
        9 & 0R10 & 1R9  \\ \hline
       10 &      & 1R11 \\ \hline
       11 & 0R12 & 1R11 \\ \hline
       12 & 1R13 & 1R12 \\ \hline
       13 & 1R14 &      \\ \hline
       14 & 0L15 &      \\ \hline
       15 & 0L16 & 1L15 \\ \hline
       16 & 0R17 & 1L16 \\ \hline
       17 & 0R18 & 1O10 \\ \hline
       18 &      & 0R19 \\ \hline
       19 &      & 1L20 \\ \hline
       20 & 1L21 & 0R25 \\ \hline
       21 & 0R22 & 1R20 \\ \hline
       22 & 0L23 & 1R22 \\ \hline
       23 &      & 0L24 \\ \hline
       24 & 0R19 & 1L24 \\ \hline
       25 & 0L26 & 1L25 \\ \hline
       26 & 0R7  & 1L26 \\ \hhline
    \end{tabularx}
    \caption{解答 5.4}
    \label{sol:5.4}
\end{table}
\end{solution}

\begin{problem}
证明二元不动点定理: 对于任何 $F, G \in \Lambda$, 存在 $X, Y \in \Lambda$, 满足
\begin{align}
F X Y & = X \\
G X Y & = Y
\end{align}
\end{problem}

\begin{solution}
消元法, 对于 (2) 式 $GXY = Y$, 有 $Y = \Theta (G X)$. 将其带入 (1) 式, 有 $F X (\Theta (G X)) = X$. 对左边进行 $\lambda$ 抽象, $(\lambda z. F z (\Theta (G z))) X = X$. 所以有解
\begin{align*}
    X = & \Theta (\lambda z. F z (\Theta (G z))) \\
    Y = & \Theta (G (\Theta (\lambda z. F z (\Theta (G z)))))
\end{align*}
\end{solution}

\begin{problem}
设机器 $M_2$ 定义如下表\ref{tab:5.6}, 对于输入 $(2,1):01^n01^m01^k00\cdots$, 其中 $n,m,k\in \mathbb{N}^+$, 求输出.
\end{problem}

\begin{table}[H]
    \centering
    \begin{tabularx}{\textwidth}{@{}Y|Y|Y@{}} \hhline
          & 0   & 1   \\ \hline
        1 & 0R2 & 0R1 \\ \hline
        2 & 1R3 & 0R1 \\ \hline
        3 & 1R4 &     \\ \hline
        4 & 1R5 &     \\ \hline
        5 & 1L6 &     \\ \hline
        6 & 0R7 & 1L6 \\ \hhline
    \end{tabularx}
    \caption{题目 5.6}
    \label{tab:5.6}
\end{table}

\begin{solution}
输入 $(2,1):01^n01^m01^k00\cdots$, 其中 $n,m,k\in \mathbb{N}^+$, 输出 $7 : 0^{n+m+k+3}\underset{\uparrow}{1}1110\cdots$, 即计算函数 $f(x,y,z)=3$.
\end{solution}

\begin{problem}
构造机器计算函数 $f(x) = \lfloor \sqrt{x} \rfloor$.
\end{problem}

\begin{solution}
根据 $\sqrt{x}$ 的定义, 使用 $\mu$ 算子可以作出 $f(x) = \mu y. [S(x) \dotminus sq(S(y))]$, 其中 $sq(x) = x^2$.

定义机器 $M_1$ 为表 \ref{sol:5.7-1} (其中 $u < v$).

\begin{table}[H]
    \centering
    \begin{tabularx}{\textwidth}{@{}Y|Y|Y@{}} \hhline
          & 0   & 1   \\ \hline
        1 &     & 0R2 \\ \hline
        2 & 0L3 & 0R5 \\ \hline
        3 & 0L3 & 1L4 \\ \hline
        4 & 0Ru & 1L4 \\ \hline
        5 & 0L6 & 0R5 \\ \hline
        6 & 0L6 & 1L6 \\ \hline
        7 & 0Rv & 1L6 \\ \hhline
    \end{tabularx}
    \caption{题目 5.7 - 1}
    \label{sol:5.7-1}
\end{table}

从而对于输入 $1 : 0\overline{x}0\overline{y}0\underset{\uparrow}{\overline{0}}0\cdots$, $M_1$ 输出 $u : 0\overline{x}0\underset{\uparrow}{\overline{y}}0\cdots$; 对于输入 $1 : 0\overline{x}0\overline{y}0\underset{\uparrow}{\overline{n}}0\cdots$ ($n > 0$), $M_1$ 输出 $v : 0\overline{x}0\underset{\uparrow}{\overline{y}}0\cdots$.

令机器 $M_2$ 为
\[
M_1 \concat \machine{S} + (v - 1) \concat \machine{shiftl} \concat \machine{$\text{copy}_2$}^2 \concat \machine{g} \concat \machine{compress}
\]
其中 $v = 8$, \machine{S} 为后继函数的机器, \machine{mul} 为题目5.3定义的乘法函数, \machine{sub} 为题目 5.11 定义的减法函数. 机器 $g$ 为
\[
\machine{shiftr} \concat \machine{S} \concat \machine{$\text{copy}_1$} \concat \machine{shiftl} \concat \machine{mul} \concat \machine{compress} \concat \machine{shiftl} \concat \machine{sub}
\]
有 $M_2 | 0\overline{x}0\overline{y}0\rwhead{\overline{0}}0\cdots \twoheadrightarrow 0\overline{x}0\rwhead{\overline{y}}000\cdots$; $M_2 | 0\overline{x}0\overline{y}0\rwhead{\overline{1}}0\cdots \twoheadrightarrow 0\overline{x}0\overline{(y+1)}0\rwhead{\overline{g(x,y+1)}}0\cdots$.

令机器 $M_3$ 为表 \ref{sol:5.7-2}.

\begin{table}[H]
    \centering
    \begin{tabularx}{\textwidth}{@{}Y|Y|Y@{}} \hhline
          & 0   & 1   \\ \hline
        1 & 1R2 & 1R1 \\ \hline
        2 & 0R2 &     \\ \hline
        3 & 1L4 &     \\ \hline
        4 & 0L4 &     \\ \hline
        5 & 0R5 & 1L4 \\ \hhline
    \end{tabularx}
    \caption{题目 5.7 - 2}
    \label{sol:5.7-2}
\end{table}

令机器 \machine{f} 为
\[
M_3 \concat \machine{$\text{copy}_2$}^2 \concat \machine{g} \concat \machine{compress} \concat \text{repeat } M_2 \concat \machine{shiftl} \concat \machine{erase}
\]
其中 $\text{repeat } M_2$ 为 $M_2[u:=1]$, 机器 \machine{f} 计算了函数 $f(x) = \lfloor \sqrt{x} \rfloor$.
\end{solution}

\begin{problem}
设机器 \machine{$f_1$} 计算函数 $f_1$, 机器 \machine{$f_2$} 计算函数 $f_2$, 这里 $f_1, f_2$ 为一元数论全函数. 构造机器 \machine{$f$} 计算函数 $f(x)=f_1(x)+f_2(x)$.
\end{problem}

\begin{solution}
$\machine{$f$} = \copymachine{1} \concat \machine{$f_1$} \concat \compressmachine \concat \shiftlmachine \concat \copymachine{2} \concat \shiftrmachine \concat \machine{$f_2$} \concat \compressmachine \concat \shiftlmachine^2 \concat \erasemachine \concat \machine{add}$.
\end{solution}

\begin{problem}
设 $f(x)=h(g_1(x), g_2(x), g_3(x))$, 试由机器 \machine{$g_1$}, \machine{$g_2$}, \machine{$g_3$} 和 \machine{$h$} 构造机器 \machine{$f$}.
\end{problem}

\begin{solution}
\begin{align*}
\machine{$f$} & = \copymachine{1} \concat \machine{$g_1$} \concat \compressmachine \concat \shiftlmachine \\
& \concat \copymachine{2} \concat \shiftrmachine \concat \machine{$g_2$} \concat \compressmachine \concat \shiftlmachine^2 \\
& \concat \copymachine{3} \concat \shiftrmachine^2 \concat \machine{$g_3$} \concat \compressmachine \concat \shiftlmachine^3 \\
& \concat \erasemachine \concat \machine{$h$}
\end{align*}
\end{solution}

\begin{problem}
证明: 对于任何 $M, N \in \Lambda$,
\[
M =_\beta N \Leftrightarrow \lambda\beta \vdash M = N
\]
\end{problem}

\begin{solution}
首先证明 ``$\Rightarrow$''. 证明, 若 $M =_\beta N$, 那么 $\lambda\beta \vdash M = N$.

先证明一个引理, 若 $M \sreduceto N$, 那么 $\lambda\beta \vdash M = N$.

对 $\sreduceto$ 的结构作归纳.
\begin{enumerate}
\item $(M, N) \in \beta$, 那么由 $\lambda\beta$ 形式系统的 $\beta$ 公理, 可知 $\lambda\beta \vdash M = N$.
\item $(M, N)$ 呈形 $(MP, MQ)$, 根据结构归纳假设若 $P \sreduceto Q$, 那么 $\lambda\beta \vdash P = Q$, 因为 $\mu$ 公理, $\lambda\beta \vdash M P = M Q$.
\item $(M, N)$ 呈形 $(PM, QM)$, 根据结构归纳假设若 $P \sreduceto Q$, 那么 $\lambda\beta \vdash P = Q$, 因为 $\upsilon$ 公理, $\lambda\beta \vdash PM = QM$.
\item $(M, N)$ 呈形 $(\lambdabstract{z}{P}, \lambdabstract{z}{Q})$, 根据结构归纳假设若 $P \sreduceto Q$, 那么 $\lambda\beta \vdash P = Q$, 因为 $\xi$ 公理, $\lambda\beta \vdash \lambdabstract{z}{P} = \lambdabstract{z}{Q}$.
\end{enumerate}
在证明, 若 $M \leftarrow_\beta N$, 那么 $\lambda\beta \vdash M = N$. 此时 $N \sreduceto M$, 所以 $\lambda\beta \vdash N = M$, 根据 $\sigma$ 公理, $\lambda\beta \vdash M = N$.

根据题目 3.9, 存在序列 $P_0, \dots, P_n \in \Lambda$, 使得 $M \equiv P_0$, $N \equiv P_n$, 且对于任何 $0 \le i < n$, $P_i \sreduceto P_{i+1}$ 或 $P_{i+1} \sreduceto P_{i}$. 所以 $\lambda\beta \vdash P_0 = P_1, \dots, \lambda\beta \vdash P_{n-1} = P_n$. 连续使用公理 $\tau$, 可知 $\lambda\beta \vdash P_0 = P_n$, 因此 $\lambda\beta \vdash M = N$.

然后证明 ``$\Leftarrow$''. 证明, 若 $\lambda\beta \vdash M = N$, 那么 $M =_\beta N$.

对 $\lambda\beta \vdash M = N$ 的证明作结构归纳.
\begin{enumerate}
\item $M = N$ 为公理 $\rho$ 或 $\beta$ 得到, 易见 $M =_\beta N$.
\item $M = N$ 由公理 $\sigma$ 得到, 此时我们有 $N = M$, 由归纳假设 $M =_\beta N$.
\item $M = N$ 由公理 $\tau$ 得到, 此时我们有 $M = N, N = L$, 有归纳假设 $M =_\beta N, N =_\beta L$, 所以 $M =_\beta L$.
\item $M = N$ 由公理 $\mu$ 或公理 $\upsilon$ 得到, 根据 $=_\beta$ 的合拍性易知成立.
\item $M = N$ 由公理 $\xi$ 得到, 根据题目 3.8 证明的结论易知成立.
\end{enumerate}

因此 $M =_\beta N$.

综上所述, $M =_\beta N \Leftrightarrow \lambda\beta \vdash M = N$.
\end{solution}

\begin{problem}
证明: 对于任何 $M, N \in \Lambda$,
\[
M =_{\beta\eta} N \Leftrightarrow \lambda\beta\eta \vdash M = N
\]
\end{problem}

\begin{solution}
首先证明 ``$\Rightarrow$''. 证明, 若 $M =_{\beta\eta} N$, 那么 $\lambda\beta\eta \vdash M = N$.

先证明一个引理, 若 $M \to_{\beta\eta} N$, 那么 $\lambda\beta\eta \vdash M = N$.

对 $\to_{\beta\eta}$ 的结构作归纳.
\begin{enumerate}
\item $(M, N) \in \beta$, 那么由 $\lambda\beta\eta$ 形式系统的 $\beta$ 公理, 可知 $\lambda\beta\eta \vdash M = N$.
\item $(M, N) \in \eta$, 那么由 $\lambda\beta\eta$ 形式系统的 $\eta$ 公理, 可知 $\lambda\beta\eta \vdash M = N$.
\item $(M, N)$ 呈形 $(MP, MQ)$, 根据结构归纳假设若 $P \to_{\beta\eta} Q$, 那么 $\lambda\beta\eta \vdash P = Q$, 因为 $\mu$ 公理, $\lambda\beta\eta \vdash M P = M Q$.
\item $(M, N)$ 呈形 $(PM, QM)$, 根据结构归纳假设若 $P \to_{\beta\eta} Q$, 那么 $\lambda\beta\eta \vdash P = Q$, 因为 $\upsilon$ 公理, $\lambda\beta\eta \vdash PM = QM$.
\item $(M, N)$ 呈形 $(\lambdabstract{z}{P}, \lambdabstract{z}{Q})$, 根据结构归纳假设若 $P \to_{\beta\eta} Q$, 那么 $\lambda\beta\eta \vdash P = Q$, 因为 $\xi$ 公理, $\lambda\beta\eta \vdash \lambdabstract{z}{P} = \lambdabstract{z}{Q}$.
\end{enumerate}
在证明, 若 $M \leftarrow_\beta N$, 那么 $\lambda\beta\eta \vdash M = N$. 此时 $N \to_{\beta\eta} M$, 所以 $\lambda\beta\eta \vdash N = M$, 根据 $\sigma$ 公理, $\lambda\beta\eta \vdash M = N$.

根据题目 3.9, 存在序列 $P_0, \dots, P_n \in \Lambda$, 使得 $M \equiv P_0$, $N \equiv P_n$, 且对于任何 $0 \le i < n$, $P_i \to_{\beta\eta} P_{i+1}$ 或 $P_{i+1} \to_{\beta\eta} P_{i}$. 所以 $\lambda\beta\eta \vdash P_0 = P_1, \dots, \lambda\beta\eta \vdash P_{n-1} = P_n$. 连续使用公理 $\tau$, 可知 $\lambda\beta\eta \vdash P_0 = P_n$, 因此 $\lambda\beta\eta \vdash M = N$.

然后证明 ``$\Leftarrow$''. 证明, 若 $\lambda\beta\eta \vdash M = N$, 那么 $M =_{\beta\eta} N$.

对 $\lambda\beta\eta \vdash M = N$ 的证明作结构归纳.
\begin{enumerate}
\item $M = N$ 为公理 $\rho$, $\beta$ 或 $\eta$ 得到, 易见 $M =_{\beta\eta} N$.
\item $M = N$ 由公理 $\sigma$ 得到, 此时我们有 $N = M$, 由归纳假设 $M =_{\beta\eta} N$.
\item $M = N$ 由公理 $\tau$ 得到, 此时我们有 $M = N, N = L$, 有归纳假设 $M =_{\beta\eta} N, N =_{\beta\eta} L$, 所以 $M =_{\beta\eta} L$.
\item $M = N$ 由公理 $\mu$ 或公理 $\upsilon$ 得到, 根据 $=_{\beta\eta}$ 的合拍性易知成立.
\item $M = N$ 由公理 $\xi$ 得到, 根据题目 3.8 证明的结论易知成立.
\end{enumerate}
因此 $M =_{\beta\eta} N$.

综上所述, $M =_{\beta\eta} N \Leftrightarrow \lambda\beta\eta \vdash M = N$.
\end{solution}

\begin{problem}
证明: 对于任何 $M, N \in \Lambda$, 若 $M =_\beta N$, 则存在 $T$ 使 $M \reduceto T$ 且 $N \reduceto T$. 这就是 $=_\beta$ 的 CR 性质.
\end{problem}

\begin{solution}
设 $M =_\beta N$, 根据题目 3.9 可知, 存在序列 $P_0, \dots, P_n \in \Lambda$, 使得 $P \equiv P_0$, $Q \equiv P_n$, 且对于任何 $0 \le i < n$, $P_i \sreduceto P_{i+1}$ 或 $P_{i+1} \sreduceto P_{i}$.

下面对 $i$ 作归纳证明, 存在 $T_i \in \Lambda$ 使得 $P_0 \reduceto T_i$ 且 $P_i \reduceto T_i$.

奠基: 当 $i=0$ 时, 取 $T_0$ 为 $M$ 即可.

归纳假设: 当 $i=k$ $(k < n)$ 时, 存在 $T_k \in \Lambda$ 使得 $P_0 \reduceto T_k$ 且 $P_k \reduceto T_k$.

归纳步骤: 当 $i = k + 1$ 时, 由归纳假设知 $P_0 \reduceto T_k$ 且 $P_k \reduceto T_k$.

情况 1: $P_k \sreduceto P_{k+1}$, 从而有 CR 性质, 存在 $T_{k+1}$ 满足 $T_k \reduceto T_{k+1}$ 且 $P_{k+1} \reduceto T_{k+1}$. 因为 $P_0 \reduceto T_k$, 所以 $P_0 \reduceto T_{k+1}$, 从而命题成立.

情况 2: $P_{k+1} \sreduceto P_k$, 于是 $P_{k+1} \reduceto T_k$, 又 $P_0 \reduceto T_k$, 所以存在 $T_{k+1} \equiv T_k$, 命题成立.

综上, 存在 $T_n$ 使得 $P_0 \reduceto T_n$ 且 $P_n \reduceto T_n$, 取 $T \equiv T_n$ 有 $M \reduceto T$ 且 $N \reduceto T$.
\end{solution}

\begin{problem}
证明: 若在形式系统 $\lambda\beta$ 中加入下述公理:
\[
(A) ~ ~ \lambdabstract{xy}{x} = \lambdabstract{xy}{y}
\]
则对于任何 $M, N \in \Lambda$, $\lambda\beta + (A) \vdash M = N$.
\end{problem}

\begin{solution}
根据公理 $A$, $\lambdabstract{xy}{x} = \lambdabstract{xy}{y}$, 对于任意 $M, N \in \Lambda$, 使用两次 $\upsilon$ 公理有 $(\lambdabstract{xy}{x}) M N = (\lambdabstract{xy}{y}) M N$.

对于左边, 使用两次 $\beta$ 公理 $(\lambdabstract{xy}{x}) M N = M$. 对于右边, $(\lambdabstract{xy}{y}) M N = N$. 然后使用公理 $\tau$, 有 $M = N$.

因此对于任何 $M, N \in \Lambda$, $\lambda\beta + (A) \vdash M = N$.
\end{solution}

\begin{problem}
设 $f : \N{} \to \N{}$ 是 Turing-可计算的, 构造机器 M 使其输出 $f$ 的最小零点.
\end{problem}

\begin{solution}
构造机器 $M_1$, 如下表 \ref{sol:5.14-1}.

\begin{table}[H]
    \centering
    \begin{tabularx}{\textwidth}{@{}Y|Y|Y@{}} \hhline
          & 0   & 1   \\ \hline
        1 &     & 0R2 \\ \hline
        2 & 0L3 & 0R5 \\ \hline
        3 & 0L3 & 1L4 \\ \hline
        4 & 0Ru & 1L4 \\ \hline
        5 & 0L6 & 0R5 \\ \hline
        6 & 0L6 & 1R7 \\ \hline
        7 & 1L8 &     \\ \hline
        8 & 0Rv & 1L8 \\ \hhline
    \end{tabularx}
    \caption{解答 5.14 $M_1$}
    \label{sol:5.14-1}
\end{table}

从而输入 $1 : 0\overline{x}0\cdots0\rwhead{\overline{y}}0\cdots$, 若 $y = 0$, $M_1$ 输出 $u : 0\rwhead{1^{x+1}}0\cdots$; 若 $y > 0$, 输出 $v : 0\rwhead{1^{x+2}}0\cdots$.

令 $M_2 = \copymachine{1} \concat \machine{$f$} \concat M_1$, 则 $M = \operatorname{repeat} M_2$ 为所求机器, 即 $M = M_2[v := 1]$, 且输出状态为 $u$.
\end{solution}

\begin{problem}
证明定理 5.21 中的函数 $g$ 为一般递归函数.
\end{problem}

\newtheorem{lemma}{引理}

\begin{solution}
\begin{lemma}
设 $n \in \N{}$, 构造机器 $M_n$ 使得输入 $\overline{x}$ 输出 $\overline{n}$, 即 $M_n | 10\rwhead{\overline{x}}0\cdots \twoheadrightarrow 0\cdots0\rwhead{\overline{n}}0\cdots$., 即 $M_n$ 计算常函数 $C(x) = n$.
\end{lemma}

\begin{proof}
构造 $n+2$ 个状态的机器 $M_n$ 先清空输入, 然后写入 $n+1$ 个 $1$. 示例机器如表 \ref{sol:5.15-1}.

\begin{table}[H]
    \centering
    \begin{tabularx}{\textwidth}{@{}Y|Y|Y@{}} \hhline
          & 0   & 1   \\ \hline
        1 & 1R2 & 0R1 \\ \hline
        2 & 1R3 &     \\ \hline
        3 & 1R4 &     \\ \hline
        \dots & \dots & \dots \\ \hline
        $n+1$ & 1L($n+2$) & \\ \hline
        $n+2$ & 0R($n+3$) & 1L($n+2$) \\ \hhline
    \end{tabularx}
    \caption{解答 5.15 $M_n$}
    \label{sol:5.15-1}
\end{table}

设 $M_n$ 的第 $i$ 行为 $r_i$. 于是 $\sharp r_1 = \pangle{1,1,4,2,0,4,1}$; 对于 $2 \le i \le n$, $\sharp r_i = \pangle{i,1,4,i+1,4,4,4}$; $\sharp r_{n+1} = \pangle{n+1,1,2,n+2,4,4,4}$, $\sharp r_{n+2} = \pangle{n+2,0,4,n+3,1,2,n+2}$.

所以 $\sharp M_n = \pangle{\sharp r_1, \dots, \sharp r_{n+2}} \in \EF$.
\end{proof}

\begin{lemma}
存在一般递归函数 $h(n,l)$ 使得对于任何机器 $M$, 有 $h(\sharp M, l) = \sharp(M+l)$.
\end{lemma}

\begin{proof}
设 $M$ 为机器, 令 $n = \sharp M$, 设 $M$ 有 $k+1$ 行, 从而 $k = (\max x \le n. P_x(n)) + 1$ 为对于 $n$ 的初等函数. 设它为 $k(n)$, 对于 $1 \le i \le k(n)$ 第 $i$ 行的编码为 $\sharp r_i = \operatorname{ep}_{i-1}(n)$ 对于 $i, n$ 为初等函数.

$M+l$ 为由在 $M$ 中将所有的状态加上 $l$ 得到的新机器, 设新机器 $M+l$ 的第 $i$ 行为 $r_i'$, 以下证明: $\sharp r_i'$ 可以由 $\sharp r_i$ 来表示, 从而 $\sharp r_i'$ 为对于 $(i,n,l)$ 的初等函数. 从而 $\sharp (M+l) = \pangle{\sharp r_0', \dots, \sharp r_k'} = \prod_{i=0}^{k(n)} P_i^{\sharp r_i'}$ 为对于 $n, l$ 的初等函数.

情况 1. $r_i$ 呈形 $(i,xyz,uvw)$ 从而 $r_i'$ 呈形 $(i+l,xy(z+l), uv(w+l))$. 因为 $\sharp r_i = \pangle{i, \sharp x, \sharp y, z, \sharp u, \sharp v, w}$, 所以有
\begin{align*}
\sharp r_i' & = \pangle{i+l, \operatorname{ep}_1(\sharp r_i), \operatorname{ep}_2(\sharp r_i), \operatorname{ep}_3(\sharp r_i) + l, \operatorname{ep}_4(\sharp r_i), \operatorname{ep}_5(\sharp r_i), \operatorname{ep}_6(\sharp r_i) + l} \\
& = (\sharp r_i) \cdot 2^l \cdot 7^l \cdot 17^l \\
& = \operatorname{ep}_{i-1}(n) \cdot (2 \cdot 7 \cdot 17)^l
\end{align*}
因此 $\sharp r_i'$ 为对于 $i,n,l$ 的初等函数, 设为 $f_1(i,n,l)$.

情况 2. $r_i$ 呈形为 $(i,xyz,RRR)$, $r_i'$ 呈形 $(i+l,xy(i+l),RRR)$. 因为 $\sharp r_i = \pangle{i,\sharp x, \sharp y, z, 4,4,4}$, 所以 $\sharp r_i' = \pangle{i+l,\sharp x, \sharp y, z+l,4,4,4}=\operatorname{ep}_{i-1}(n) \cdot (2\cdot 7)^l$ 为对于 $i,n,l$ 的初等函数, 设为 $f_2(i,n,l)$.

情况 3. $r_i$ 呈形为 $(i,LLL,uvw)$, $r_i'$ 呈形 $(i+l,LLL,uv(w+l))$. 类似情况 2, 可以知道 $\sharp r_i' = \operatorname{ep}_{i-1}(n) \cdot (2 \cdot 17)^l$ 为对于 $i,n,l$ 的初等函数, 设为 $f_3(i,n,l)$.

所以 \[
\sharp r_i' = \begin{cases}
f_3(i,n,l) & \text{if } \operatorname{ep}_1(\sharp r_i) = 2 \\
f_2(i,n,l) & \text{if } \operatorname{ep}_4(\sharp r_i) = 4 \\
f_1(i,n,l) & \text{o.w.}
\end{cases}
\]
易见, $\sharp r_i'$ 为对于 $i,n,l$ 的初等函数, 设为 $f(i,n,l)$.

因为 $\sharp (M+l) = \prod_{i=0}^{k(n)}P_i^{f(i,n,l)}$, 所以令 $h(n,l) = \prod_{i=0}^{k(n)} P_i^{f(i,n,l)}$. 易见 $h(\sharp M, l) = \sharp (M+l) \in \EF$.
\end{proof}

\begin{lemma}
设 $n = \sharp M$, $M_n$ 为引理 1 定义的机器, 令 $\hat{M} = M_n \concat M$, $j(n) = \sharp \hat{M} \in \EF$.
\end{lemma}

\begin{proof}
$\hat{M} = M_n \concat M$ 即为 $M_n + (M + (n+2))$, 从而由以上引理可知: $M_n$ 有 $n+3$ 行 $r_0, r_1, \dots, r_n+2$; $M+(n+2)$ 有 $k(n)+1$ 行 $r_0', r_1', \dots, r_{k(n)}'$.

从而 \begin{align*}
\sharp \hat{M} & = \pangle{\sharp r_1, \dots, \sharp r_{n+2}, \sharp r_1', \dots, \sharp r_{k(n)}'} \\
& = P_0^{\sharp r_1} \cdot \prod_{i = 2}^n P_{i-1}^{\pangle{i,1,4,i+1,4,4,4}} \cdot P_n^{\sharp r_{n+1}} \cdot P_{n+1}^{\sharp r_{n+2}} \cdot \prod_{j=1}^{k(n)} P_{n+1+i}^{\sharp r_j'} \\
& = P_0^{\pangle{1,1,4,2,0,4,1}} \cdot \prod_{i=2}^n P_{i-1}^{\pangle{i,1,4,i+1,4,4,4}} \cdot P_{n}^{\pangle{n+1,1,2,n+2,4,4,4}} \cdot P_{n+1}^{\pangle{n+2,0,4,n+3,1,2,n+2}} \cdot \prod_{j=1}^{k(n)} P_{n+1+j}^{f(i,n,l)} \in \EF
\end{align*}

故 $j(n) = \sharp \hat{M} \in \EF$.
\end{proof}

\begin{lemma}
$S = \{ \sharp M ~ | ~ M \text{ 为机器} \}$ 是可判定的.
\end{lemma}

\begin{proof}
$n \in S$ $\Leftrightarrow$ 有机器 $M$ 使得 $n = \sharp M$.

$\Leftrightarrow$ 有机器 $M$ 有 $k(n)$ 行 $r_0, r_1, \dots, r_{k(n)}$ 且 $n = \pangle{\sharp r_1, \dots, \sharp r_{k(n)}}$, 其中 $r_i$ 为机器 $M$ 的第 $i$ 行, 而 $\operatorname{ep}_{i-1}(n)$ 为机器第 $i$ 行的编码.

又因为 $m$ 为某机器行的编码 (记为 $\operatorname{codel}(m,n)$) 

$\Leftrightarrow$ $m = \sharp r_i$ 且 $i \le k(n)$.

$\Leftrightarrow$ $m = \sharp \machine{$i$ | $xyz$ | $uvw$}$ 且 $i \le k(n)$.

$\Leftrightarrow$ \begin{align*}
\{[(\operatorname{ep}_0(m) \le k(n)) \land & (\operatorname{ep}_1(m) \in \{0,1\}) \land (\operatorname{ep}_2(m)\in\{\sharp L, \sharp O, \sharp R\}) \land \\ 
& (\operatorname{ep}_4(m) \in \{0,1\}) \land (\operatorname{ep}_5(m)\in\{\sharp L, \sharp O, \sharp R\})]\} \\
\lor \{[(\operatorname{ep}_0(m) \le k(n)) \land & (\operatorname{ep}_1(m) \in \{0,1\}) \land (\operatorname{ep}_2(m)\in\{\sharp L, \sharp O, \sharp R\}) \land \\ 
& (\operatorname{ep}_4(m) = \operatorname{ep}_5(m) = \operatorname{ep}_6(m) = \sharp R) ]\} \\
\lor \{[(\operatorname{ep}_0(m) \le k(n)) \land & (\operatorname{ep}_4(m) \in \{0,1\}) \land (\operatorname{ep}_5(m)\in\{\sharp L, \sharp O, \sharp R\}) \land \\ 
& (\operatorname{ep}_1(m) = \operatorname{ep}_2(m) = \operatorname{ep}_3(m) = \sharp R) ]\} \\
\end{align*}
所以 $\operatorname{codel}(m,n)$ 为初等数论谓词, 从而
\[
n \in S \Leftrightarrow (\forall i \le k(n)). [\operatorname{codel}(ep_{i-1}(n), n)] \land k(n) = (\max x\le n. P_x | n) + 1
\]
所以 $n \in S$ 为初等数论谓词, 所以 $\chi_S \in \EF$, 从而 $\chi_S$ 为 recursive, 故 $S$ 可判定.
\end{proof}

回到原命题.

因为 $j(n) = \sharp \hat{M} \in \EF$ (引理 3), $\chi_S(n) \in \EF$ (引理 4).

所以 $g(n) = \begin{cases}j(n) & \text{if } n = \sharp M \\ 0 & \text{o.w.}\end{cases} = \begin{cases}j(n) & \text{if } n \in S \\ 0 & \text{o.w.}\end{cases} = j(n) \cdot N(\chi_S(n)) \in \EF$.

\end{solution}

\begin{problem}
证明引理 5.25 中的函数 $e(m,l)$ 为一般递归函数.
\end{problem}

\begin{solution}
令 $m = \sharp M$, $l = \sharp_t A$, 从而先定义 3 个关于 $l$ 的函数, $j=j(l)=(l)_0$, $k=k(l)=(l)_1$, $a_j = a_j(l) = \begin{cases}1 & \text{if } (l)_{j+1} = 1 \\ 0 & \text{if } (l)_{j+1} = 2\end{cases} = 2 \dotminus (l)_{j+1}$.

易见 $j(l), k(l), a_j(l)$ 皆为初等函数.

令 $M$ 的第 $i$ 行为 \machine{$j$ | $xyz$ | $uvw$}, 从而 $\sharp x = \sharp x(m,l)$, $\sharp y = \sharp y(m,l), \dots, \sharp w = \sharp w(m,l)$ 皆为关于 $m,l$ 的初等函数.

因为 $M(A)$ 有定义 $\Leftrightarrow$ $(j,k) : A$ 关于 $M$ 有后继 $\Leftrightarrow$ $(a_j,k) \in \operatorname{Dom}(M)$ 且 $j+p(a_j,k) \ge 1$ $\Leftrightarrow$ \begin{align*}
     & [a_j = 0 \to ((x,y,z \text{ 不皆为 } L) \land (y = L \to j - 1 \ge 1))] \\
\lor & [a_j = 1 \to ((u,v,w \text{ 不皆为 } R) \land (v = L \to j - 1 \ge 1))] \\
\end{align*}
所以, $M(A)$ 有定义为 $(m,l)$ 的初等谓词. 故其特征函数 $\chi(m,l) \in \EF$.

函数 $d(m,l) = N(\chi(m,l))$ 为初等的, 下面作 $e(m,l)$:

因为 \begin{align*}
\sharp d(a_j,k) & = \begin{cases}\sharp x & a_j = 0 \\ \sharp u & a_j = 1 \end{cases} = \sharp x \cdot N(a_j) + \sharp u \cdot N^2(a_j) \\ 
\sharp p(a_j,k) & = \sharp y \cdot N(a_j) + \sharp v \cdot N^2(a_j) \\
\sharp s(a_j,k) & = \sharp z \cdot N(a_j) + \sharp w \cdot N^2(a_j) \\
\end{align*}

所以令 $e(m,l) = d(m,l) \times \pangle{\sharp d(a_j,k), \sharp p(a_j,k), \sharp s(a_j, k)} \in \EF$.
\end{solution}

\begin{problem}
令 $S = \{ \sharp M ~ | ~ M \text{ 为 Turing 机} \}$, 证明 $S$ 为 Turing-可计算.
\end{problem}

\begin{solution}
见题目 5.15 中的引理 4.
\end{solution}

\begin{problem}
令 $D \equiv \lambdabstract{xyz}{z(Ky)x}$, 证明: 对于任意的 $X, Y \in \Lambda$,
\begin{align*}
DXY\numeral{0} & = X \\
DXY\numeral{n+1} & = Y \\
\end{align*}
这里 $K \equiv \lambdabstract{xy}{x}$, $\numeral{n} \equiv \lambdabstract{fx}{f^nx}$.
\end{problem}

\begin{solution}

第一, $D X Y \numeral{0} \equiv (\lambdabstract{xyz}{z(Ky)x}) X Y \numeral{0} \reduceto \numeral{0} (K Y) X = (\lambdabstract{fx}{x}) (K Y) X \reduceto X$.

第二, $D X Y \numeral{n+1} \equiv (\lambdabstract{xyz}{z(Ky)x}) X Y \numeral{n+1} \reduceto \numeral{n+1} (K Y) X = (\lambdabstract{fx}{f^{n+1}x}) (K Y) X \reduceto ((K Y)^{n+1} X) = ((\lambdabstract{y}{Y})^{n+1} X) \reduceto Y$.
\end{solution}

\begin{problem}
设 $\operatorname{Exp} \equiv \lambdabstract{xy}{yx}$, 证明: 对于任意的 $n \in \mathbb{N}$ 和 $m \in \mathbb{N}^*$,
\[
\operatorname{Exp} \numeral{n} \numeral{m} =_\beta \numeral{n^m}
\]
($\operatorname{Exp}$ 由 Rosser 教授作出)
\end{problem}

\begin{solution}
对 $m$ 进行数学归纳.

奠基: 当 $m = 1$, 
\begin{align*}
\operatorname{Exp} \numeral{n} \numeral{1} & = (\lambdabstract{xy}{yx}) \numeral{n} (\lambdabstract{fx}{fx}) \\
& = (\lambdabstract{fx}{fx}) \numeral{n} \\
& = \lambdabstract{x}{\lambdabstract{fz}{f^nz} x} \\
& = \lambdabstract{xz}{x^n z} \\
& \equiv \numeral{n^1}
\end{align*}
成立。

归纳假设: 当 $m = k$ 时, $\operatorname{Exp} \numeral{n} \numeral{k} =_\beta \numeral{n^k}$.
\begin{align*}
\operatorname{Exp} \numeral{n} \numeral{k} & = (\lambdabstract{xy}{yx}) \numeral{n} (\lambdabstract{fx}{f^kx}) \\
& = (\lambdabstract{fx}{f^kx}) \numeral{n} \\
& = \lambdabstract{x}{(\lambdabstract{fz}{f^nz})^k x}
\end{align*}
因此 $\lambdabstract{x}{(\lambdabstract{fz}{f^nz})^k x} = \numeral{n^k}$.

归纳步骤: 对于 $m = k + 1$,
\begin{align*}
\operatorname{Exp} \numeral{n} \numeral{k+1} & = (\lambdabstract{xy}{yx}) \numeral{n} (\lambdabstract{fx}{f^{k+1}x}) \\
& = (\lambdabstract{fx}{f^{k+1}x}) \numeral{n} \\
& = \lambdabstract{x}{(\lambdabstract{fz}{f^n z})^{k+1} x} \\
& = \lambdabstract{x}{(\lambdabstract{fz}{f^n z})  ((\lambdabstract{fz}{f^n z})^k x)} \\
& = \lambdabstract{x}{(\lambdabstract{fz}{f^n z}) (\lambdabstract{y}{((\lambdabstract{fz}{f^n z})^k y)}) x} \\
& = \lambdabstract{x}{(\lambdabstract{fz}{f^n z}) (\numeral{n^k} x}) & (\text{By I.H.}) \\
& = \lambdabstract{xz}{(\lambdabstract{y}{x^{n^k} y})^n z} \\
& = \lambdabstract{xz}{(\lambdabstract{y}{x^{n^k} y}) (\dots ((\lambdabstract{y}{x^{n^k} y}) z))} \\
& \equiv \lambdabstract{xz}{x^{n^{k+1}}z} \\
& \equiv \numeral{n^{k+1}}
\end{align*}

综上, 对于任意的 $n \in \mathbb{N}$ 和 $m \in \mathbb{N}^*$, $\operatorname{Exp} \numeral{n} \numeral{m} =_\beta \numeral{n^m}$.
\end{solution}

\begin{problem}
\begin{enumerate}
\item 什么是通用 Turing 机 (universal Turing machine)?
\item 通用 Turing 机起什么作用?
\end{enumerate}
\end{problem}

\begin{solution}
\begin{enumerate}
\item 通用 Turing 机 (universal Turing machine) 指的是机器 $U$ 使对任何机器 $M$ 和任何 $(n_1, n_2, \dots, n_k) \in \N{k}$, 满足
\[
M | \overline{(n_1, n_2, \dots, n_k)} \twoheadrightarrow \overline{y} \Leftrightarrow U | \overline{(\sharp M, n_1, n_2, \dots, n_k)} \twoheadrightarrow \overline{y}
\]
\item 通用 Turing 机单凭借自身就可以完成任何 Turing 机可能做到的任何事. 通用性是指这样的机器能模拟任何其它 Turing 机. 通用 Turing 机载早期程序储存式计算机的研制中起到了重要的促进作用.
\end{enumerate}
\end{solution}

\begin{problem}
什么是 Church-Turing Thesis?
\end{problem}

\begin{solution}
直觉能行可计算 (部分) 函数类等同于 Turing-可计算 (部分) 函数类.
\end{solution}

\end{document}